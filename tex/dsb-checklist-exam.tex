\section{Checklist for the exam}\label{sec:checklist-exam}

The following list is incomplete but it should give you some idea on what to focus on when preparing for the exam:

\itex{
\item I have read the lecture notes from page 1 to page \pageref{page:last-page}.
\item I have \R and \Rstudio installed on my PC. Both is ready to use.
\item I know how to install and load packages.
\item I know how to set a working directory on my PC.
\item I know how to import comma seperated data to \R.
\item I know how to create variables in \R and how to combine variables into a data frame.
\item I know how to assign names in \R to certain pieces of data.
\item I know how to work with data frames in \R.
\item I have installed all packages mentioned in the lecture notes and in the exercises, respectively.
\item I have gone through all swirl learning modules of the swirl course \textit{swirl-it}.
\item I have good knowledge on how to use the extract operator and the pipe operator.
\item I know how to write a script in \R and I know how to \textit{run} all lines of code of the script.
\item I know the \textit{dplyr} package and how to use its functions (filter, select, arrange, summarise, mutate).
\item I know the \textit{ggplot2} package and how to create simple graphs with ggplot.
\item I know how to run a regression with \R, show the output and interpret the results.
\item I know how to get descriptive statistics from a dataset and how to graphically visualize the most important statistics.

}