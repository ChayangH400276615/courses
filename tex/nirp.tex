


%\paragraph{Do Negative Interest Rates Work?}
\begin{figure}
	\begin{center}
		\includegraphics[width=.5\linewidth]{../../../pic/badly}
	\end{center}
\caption{How do negative interest rates work? Badly.}
	\note{Source: https://app.hedgeye.com/insights/65863-the-folly-of-negative-nominal-interest-rates}
\end{figure}




\boxx{
\paragraph{Learning objectives} 
Students will learn to explain and evaluate the goals and the effectiveness of a monetary policy with negative interest rates policy (NIRP).\medskip

\paragraph{Outline of the lecture}	
In this short lecture, I will thrive the following questions:
\itex{
	\item What is the objective of ECB's NIRP?
	\item How can ECB's NIRP effect economies?
	\item What is so special on a NIRP from a macroeconomic perspective?
	\item Can negative central bank policy rates transmit into negative deposit rates or into negative lending interest rate (the cost of debt for the borrower)?
	\item The key question for macroeconomic monetary policy (and this lecture) is therefore whether lowering interest rates below zero can be an effective tool for stimulating aggregate demand.
	\item What are the cons of a NIRP?
}

\paragraph{Abstract}	In recent years, the European Central Bank (ECB) and some other central banks engaged in a radical new policy experiment by setting negative policy rates. This course discusses whether negative interest rates policy (NIRP) is an effective tool for stimulating aggregate demand and hence keep prices at a stable inflation of 2\%. In particular, I present aggregate and bank-level data to document that once the policy rate turns negative the pass-through to deposit and lending rates seems to be limited. 
}


\section{My stamps taught me a lot}

\begin{center}
\includegraphics[width=.28\linewidth]{../pic/german_2mio}
\includegraphics[width=.35\linewidth]{../pic/stamp_ezb}
\includegraphics[width=.35\linewidth]{../pic/now_stamp_ezb}
\captionof{figure}{Stamps and inflation}\label{fig:stamp}

%		\note{Source: }
\end{center}

I was about 10 years old and passionate about collecting stamps when I discovered an interesting stamp in an album I had received as a gift. Curious, I asked my father if I was now a double millionaire (see \autoref{fig:stamp}). Unfortunately, he had to disappoint me. 
\paragraph{Lesson 1:} Currencies can disappear and their value over time is not fixed.

Starting in 2002, Deutsche Post refused to accept all my mint stamps denominated in pfennigs. The ECB stamp is denominated in pennies and cannot now be used.
\paragraph{Lesson 2:} Currency reforms can occur without war or hyperinflation, and assets can be lost with them. 

Even if the ECB stamp were still accepted, its current value in pennies would not be enough to mail a letter. Today, it costs 156 pfennigs, or 80 cents, to send a letter. That means sending a letter has become 40 percent more expensive. 
\paragraph{Lesson 3:} Inflation still exists.


\section{Put money in a bank account and get rich}

\begin{figure}
	\begin{center}
		\includegraphics[width=.9\linewidth]{../pic/zero}
	\end{center}
	\note{Source: Refinitiv; \textcopyright FT, see: https://www.ft.com/content/82c4d584-bce3-11e9-89e2-41e555e96722}
	\caption{Five negative central bank policy rates}\label{fig:fivenegative}
\end{figure}

My parents always told me: ``Stop buying that many stamps. Better take your money to a bank and get rich from the deposit interest.'' 
I doubt, however, if this strategy is actually a valid strategy in times when the ECB and some other central banks set negative key interest rates, see \autoref{fig:fivenegative}. 
Negative interest rate policies were long considered unthinkable because economists believed in the so-called \textit{Zero Lower Bound (ZLB)}. This refers to the widely held notion that a NIRP is ineffective because private banks cannot charge negative deposit rates because people would otherwise withdraw their cash and keep it privately. As some have recently learned, there are indeed some banks that have charged negative interest rates on customer deposits without any bank runs. 

It must be said, however, that commercial banks charge negative interest rates only on a small scale. Most banks charge negative interest rates only on part of the deposited assets, only for companies or only for (short-term) assets above a certain amount. Accordingly to a survey of Verivox (see \url{https://www.verivox.de/geldanlage/themen/negativzinsen/}) from 2022 June 07:
\itex{
	\item 451 banks have published negative interest rates for retail customers on their websites or in their online price lists.
\item 23 banks charge fees for the usually free overnight deposit account. This creates a de facto negative interest rate.
\item According to media reports, some banks and savings banks charge negative interest rates but do not publish them online.
}
%
%%\paragraph{115 German banks have negative deposit rates}
%%\begin{center}
%%\includegraphics[width=.9\linewidth]{../pic/b0}\label{fig:zero}
%%\end{center}
%%\note{Source: https://www.biallo.de/geldanlage/ratgeber/so-vermeiden-sie-negativzinsen}




\section{European Central Bank}

\subsection{Objective}
The primary objective of the European Central Bank (ECB) is to maintain price stability within the Eurozone, that is, an inflation\footnote{Inflation is an increase in the general price level of goods and services. When there is inflation in an economy, the value of money decreases because a given amount will buy fewer goods and services than before.} of under 2\%. In particular, the goal is to have ``a year-on-year increase in the Harmonised Index of Consumer Prices (HICP)\footnote{The HICP is a weighted average of consumer price indices of member states who have adopted the euro.} for the euro area of below 2\%''\footnote{see: https://www.ecb.europa.eu/mopo/strategy/pricestab/html/index.en.html}. Moreover, the ECB ``shall support the general economic policies in the Union'' (see Article 127(1) TFEU).

In \autoref{fig:eurohicp}, I present inflation rates and inflation expectations from 1990 onwards. The figure shows, inflation rate circulated more or less around the 2\% and that this holds, although the Eurozone was hit by some serious economic turbulences in the last decades. Thus, we can say that the ECB was successful in achieving their main objective. 

%The question now is: How did they do it?


\begin{figure}
	\begin{center}
		\includegraphics[width=.9\linewidth]{../pic/inflation_full}
	\end{center}
\caption{Euro area HICP inflation and inflation expectations}\label{fig:eurohicp}
	\note{Source: ECB, Consensus Economics, Thomson Reuters, ECB calculations.}
\end{figure}

\subsection{Monetary policy instruments}
In recent years, the ECB has deployed an innovative, multi-pronged approach in the design of its policy stance. In a nutshell, the current policy mix basically includes four elements: (i) pushing the policy rate into negative territory, (ii) forward guidance on the future policy path, (iii) the asset purchase programmes, and (iv) the targeted longer-term refinancing operations (TLTROs). Importantly, these measures work as a package, with significant complementarities across the different instruments. However, we do not aim discuss these instruments here. Instead, we focus on the most recognized instrument, that is, the central bank interest/policy rate (CBPR). 

\subsection{Central bank interest/policy rate (CBPR)}

When people talk about the \textbf{central bank interest/policy rate (CBPR)} they usually refer to the deposit facility rate, which is actually one of the three interest rates\footnote{The other two are called marginal lending facility rate, main refinancing operation rate.} the ECB sets every six weeks. The rate defines the interest banks receive for depositing money with the central bank overnight. 


\paragraph{The textbook economics of the CBPR}
All major introductory macroeconomic textbooks assume that investments increase when interest rates decrease and vice versa. This \textit{investment function} is shown in panel (a) of the \autoref{fig:inviszero}. An increase of investments transmits into higher aggregate demand and output through the IS curve as illustrated in Mankiw's macroeconomics textbook.\footnote{For simplicity, I refrain from discussing the LM curve which closes the IS-LM model and  is used to determined equilibrium in the money market.} It doesn't need a formal model to understand that higher aggregate demand and output yields higher prices.

\begin{figure}
	\begin{center}
		\includegraphics[width=1\linewidth]{../pic/mankiw}
	\end{center}
	\note{Source: \cite{Mankiw2010Macroeconomics}: Macroeconomics, 7th Edition, p. 299.}
	\caption{The investment function and and the IS curve when interest rate is zero}\label{fig:inviszero}
\end{figure}


However, most textbooks refrain from discussing how the investment function behaves when interest rates become close to zero or even negative. The reason is that NIRP was unthinkable for a long time as a quote of Warren Buffet shows:

\begin{quotation}
``Whats happened with interest rates is really extraordinary. I mean, you can go back and read everything, Keynes, Adam Smith, Ricardo, Galbraith, Paul Samuelson, or you name. You won't see a word in my view anything I've ever seen about sustained negative interest rates. I mean, we are doing something the world hasn't seen.'' Warren Buffet in an CNBC interview 2016 (https://www.youtube.com/watch?v=q862lngj034)
\end{quotation} 


Instead, the so-called \textbf{Zero Lower Bound (ZLB)} was dominant. It says that a short-term nominal interest rate of zero cause a liquidity trap and limit the capacity of central banks to stimulate economic growth. The main argument is that it is difficult to encourage investors to deposit money at zero or negative interest rates. Thus, economist thought that central banks will not reduce interest rates below this bound. As already mentioned, they were wrong. 

\boxb{We should re-investigate the investment function. As NIRP exists, we should discuss whether NIRP is an effective tool for stimulating investments and increase prices.} 


Basically, four ways exist how NIRP can increase investments:
\enux{\item Banks can lend more to households and companies, rather than holding on to cash, which has now become costly.
\item Businesses can invest more, as funding investment is now cheaper.
\item Households could save less, or borrow to spend more.
\item Demand for the currency could fall. This might lead to a depreciation of the currency, an increase in the price of imported goods and growing demand for the country’s exports which are now cheaper for foreign buyers.
}\smallskip

If we abstract from the fourth way, the effectiveness of NIRP hinges on whether the policy rates pass-through to deposit and lending rates. 
In the next box, I will present the work of \cite{Eggertsson2017Are} to answer that question. This was one of a few studies available at the time of writing this notes, see \autoref{fig:nirpdark}.

\begin{figure}
	\begin{center}
		\includegraphics[width=.5\linewidth]{../pic/advise}
		\label{fig:advise}
	\end{center}
	\caption{Research on NIRP is a growing but underdeveloped literature strand}\label{fig:nirpdark}
	\note{Source: https://www.investmentnews.com/article/20160221/FREE/302219997/what-happens-if-the-fed-goes-negative}
\end{figure}


\paragraph{Limited Pass-Through to Deposit Rates}
In \autoref{fig:deposit}, I plot aggregate deposit rates for six economic areas in which the policy rate is negative. The red vertical lines mark the month in which policy rates became negative. The Swedish central bank lowered its key policy rate below zero in February 2015. Deposit rates, which in Sweden are usually below the policy rate, did not follow the central bank rate into negative territory. Instead, deposit rates for both households and firms remain stuck at, or just above, zero. A similar picture emerges for Denmark who crossed the ZLB twice. As was in Sweden, the negative policy rate has not been transmitted to deposit rates.
The deposit rates of Switzerland  and Japan were already very low for some time when the ZLB was passed. However, even then the other rates did not follow the policy rate into negative territory.

The ECB hit the ZLB in June 2014. As aggregate deposit rates are high in the Euro Area and therefore have more room to fall before reaching the zero lower bound. Moreover, the deposit rate does not normally follow the policy rate as closely as in the other cases. One reason for this is the heterogeneity of the Euro Area. Looking at Germany only, a similar pattern emerges as in the other cases. That is, despite negative policy rates, the deposit rate appears bounded by zero. 

\boxb{Limited Pass-Through to Deposit Rates. Overall, the presented aggregate evidence indicates that the impact of policy rates on deposit rates was limited and is strongly suggestive of a lower bound on deposit rates }

\begin{figure}
	\begin{center}
		\includegraphics[width=1\linewidth]{../pic/deposit}
	\end{center}
	\caption{Aggregate deposit and policy rates}	
	\label{fig:deposit}
	\note{Source: \cite{Eggertsson2017Are}.}
\end{figure}


\paragraph{Limited Pass-Through to Lending Rates}
In \autoref{fig:lending}, I plot lending rates for the six economic areas with negative policy rates. While lending
rates usually follow the policy rate closely, there appears to be a disconnect once the policy
rate breaks the ZLB. Lending rates in Sweden, Denmark and Switzerland seem less sensitive to the respective policy rates once they become negative. 
The Euro Area is somewhat of an outlier, as lending rates appear to have decreased. This is not surprising in light of the higher-than-zero deposit rates I document above. Again, for the case of
Germany, in which the zero lower bound on the deposit rate is binding, lending rates appear
less responsive.

\begin{center}
\includegraphics[width=1\linewidth]{../pic/lending}
\captionof{figure}{Aggregate lending and policy rates}\label{fig:lending}
\note{Source: \cite{Eggertsson2017Are}.}
\end{center}

To investigate the behavior of banks with respect to lending rates further, I present daily bank-level data for thirteen Swedish credit institutions in \autoref{fig:ilending}.
In the left panel, the bank-level mortgage rates for thirteen banks or credit institutions are plotted. The first two dashed lines capture policy rate reductions in positive territory. On both occasions, there is an immediate and homogeneous decline in bank lending rates. The same holds true when the policy rate passes the ZLB (solid line). Around the three proceeding dashed lines, which represent reductions in negative territory, the bank lending rates are strikingly different: While there is some initial reduction in lending rates, most of the rates increase again shortly thereafter. Thus, the total impact on lending rates seems to be limited. Moreover, there is a substantial increase in dispersion, with several banks keeping their lending rate roughly unchanged despite repeated interest rate reductions below zero. 

In the right panel the minimum and maximum bank lending rate are plotted, along with the policy rate (the dashed black line). The increase in dispersion after the policy rate turned negative is clearly visible. It is remarkably that the minimum bank lending rate has stayed constant since the first quarter of 2015, despite three policy rate reductions in negative territory.


\boxb{Limited Pass-Through to Lending Rates. 	While the evidence is not as clear as for deposit rates, the impact of policy rates on lending rates seems to be limited once the ZLB has passed. Moreover, the dispersion of lending rates within a territory was increased when the policy rate was negative and some banks even increased (!) their lending rates when the policy rate was negative.}

\begin{center}
\includegraphics[width=.9\linewidth]{../pic/ilending}
\captionof{figure}{Bank Level Lending Rates Sweden}\label{fig:ilending}
\note{Source: \cite{Eggertsson2017Are}.}
\end{center}


\paragraph{Does NIRP work?}
Overall, there is a lack of research that evaluates the impact of NIRP empirically. Considering the fact that NIRP is a rather new phenomena, that is not a big surprise. Moreover, the NIRP is hard to evaluate as there is no counterfactual and the policy rate instrument is just one out of many monetary instruments that have been used by the ECB in recent years. Thus, it is hard to find an empirical identification strategy that allows a sort of ceteris paribus interpretation.

The ECB was heavily criticized for the NIRP. While I cannot list all points of scepticism, I'd like to mention some reasonable ones here: 
\itex{
\item Private banks cannot pass through negative interest rates to their customer. That lowers the profitability of private banks which are likely to raise lending rates to save their interest margin. We saw this in Sweden where banks increased rates for mortgages and loans to stay profitable. Of course, this weakens policy effectiveness. 
\item If lending rates become very low, more risky investments get financed which, in turn, brings economic uncertainties to the market.
\item Finally, a NIRP may harm the reputation of the Euro because people may flee into other (financial) assets or alternative currencies like bitcoins.
}



\exex{Sparkassen CEO criticizes the ECB policy}{
Printed below, you find an open letter from the Sparkassen-CEO Helmut Schleweis to the ECB president Mario Draghi published in the BILD Zeitung a week ago. Read it and try to answer or discuss the following questions:
\enux{
\item Name the main objective of the ECB.
\item Discuss what Mr. Schleweis wants the ECB to take care of.
\item Comment on the following statement ``Anyone who invests money with you must even pay something.'' In particular, discuss who `invests' money in the ECB.
\item High interest rates are ususally good for those who have money and bad for those who want to borrow money, right? Moreover, believing in the words of Mr. Schleweis, those who have money can flee `into real estate with their money' in cases where interest rates are low. Having that in mind, isn't it that those who have money always win? 
\item The ECB set negative interest rates primarily to increase aggregate demand and prices. Discuss if Europe would `move closer' or if some states would `reduce their debts' faster with high interest rates and deflation.
}

%\begin{wrapfigure}{l}{5cm}
\begin{center}
	\includegraphics[width=0.3\textwidth]{../pic/bild}
	\note{Bild Zeitung (15.08.2019), \url{http://bit.ly/33wUw61}}
\end{center}
%\end{wrapfigure}
\zitat{
Dear Mario Draghi,

First and foremost, allow us to state that we have great respect for the difficult task you face in keeping the euro stable and Europe united.

However, what you are doing is wrong. For years, you have been throwing more and more money at the market. You have abolished the interest rate. And you have loaned money in unimaginable dimensions to states that are highly indebted.

By doing this, you are gradually changing Europe, Germany, and the lives of millions of people – not for the better, but, in the long term, for the worse. It no longer costs anything to take on debts. Saving money no longer generates any interest. Anyone who invests money with you must even pay something.

You are thereby turning the rules of the economy upside down. People who are able to are fleeing into real estate with their money – prices and rents for which are on the rise. The retirement provisions of millions of people are melting like snow in the sun. Social insurances, pension funds, and foundations are all losing great amounts of money every day, and thus losing their capacities. For years, we have taught Germany’s children that saving money makes sense, because one has to provide for bad times during crises. You are undermining that culture. All of this cannot end well in the long term.

And what is it all for? Have European states that are in a crisis used the bought time to reduce their debts? Has Europe moved closer together? Your monetary policy has achieved none of this.

If one is in a dead-end street, one should not increase the speed. It is time to turn around – step by step. Now!

Best wishes,
Helmut Schleweis
}
}
%\end{quotation}

