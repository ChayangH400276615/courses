\chapter{Sample}\label{sec:sampling}

\boxx{\textbf{In this chapter, we learn\dots}
	\itex{
		\item \dots what characterizes a \textit{good} sample, i.e., a sample that can be used for empirical analysis using statistical inference.
		\item \dots to distinguish between a sample and a population.
		\item \dots to identify biased samples.
		\item \dots to distinguish between random sampling and random assignment.
		\item \dots that a simple random sample is the gold standard because of its properties.
		\item \dots different ways to collect data and draw a sample, respectively (stratification, clustering, opportunity)
}}

In statistics and quantitative research methodology, a \textbf{sample} is a set of individuals or objects collected or selected from a statistical population by a defined procedure. The elements of a sample are known as sample points, sampling units or observations. 

Typically, the population is very large, making a census or a complete enumeration of all the individuals in the population either impractical or impossible. The sample usually represents a subset of manageable size. Samples are collected and statistics are calculated from the samples, so that one can make inferences or extrapolations from the sample to the population. 

\section{The \cite{Hite1976Hite} Report}
%In 1976, \textit{The Hite Report} is published and instantly becomes a best seller, confronting the mainstream with research-based evidence that women like to do three things:
%\enux{
%\item read
%\item masturbate, and 
%\item read about masturbating.
%} 
\begin{minipage}[t]{0.5\textwidth}
\begin{center}
	\includegraphics[width=.9\linewidth]{$HOME/Dropbox/hsf/pic/hite}
\end{center}
	%\bibentry{Hite1976Hite}
\end{minipage}
\begin{minipage}[t]{0.5\textwidth}
\begin{center}
	\includegraphics[width=.99\linewidth]{$HOME/Dropbox/hsf/pic/destat/hitereport}
	\note{Source: \url{www.theparisreview.org/blog/2017/07/21/great-moments-literacy-hite-report}}
	
\end{center}
\end{minipage}

The picture of womens' sexuality in \cite{Hite1976Hite} was probably a bit biased as the sample can hardly be considered to be a \textbf{random and unbiased} one:
\itex{
	\item Less than 5\% of all questionnaires which were sent out were filled out and returned (response bias). 
	\item The questions were only sent out to women's organizations (an opportunity sample). 
}
Thus, the \textit{results} were based on a sample of women who were highly motivated to answer survey's questions, for whatever reason.




\section{Sampling}
In statistics, we often rely on a \textbf{sample} --- that is, a small subset of a larger set of
data --- to draw inferences about the larger set. The larger set is known as the
\textbf{population} from which the sample is drawn.

Researchers adopt a variety of sampling strategies. The most straightforward is
\textbf{simple random sampling}. Such sampling requires every member of the population
to have an equal chance of being selected into the sample. In addition, the selection
of one member must be independent of the selection of every other member. That
is, picking one member from the population must not increase or decrease the
probability of picking any other member (relative to the others). In this sense, we
can say that simple random sampling chooses a sample by pure chance. To check
your understanding of simple random sampling, consider the following example.
What is the population? What is the sample? Was the sample picked by simple
random sampling? Is it biased?

\begin{center}\includegraphics[width=.7\linewidth]{$HOME/Dropbox/hsf/pic/destat/sample1}
	
	Source: \citet[p. 92]{Gonick1993Cartoon}\end{center}


\begin{center}\includegraphics[width=.6\linewidth]{$HOME/Dropbox/hsf/pic/destat/sample2}
	
	Source: \citet[p. 92]{Gonick1993Cartoon}\end{center}


\subsection{Random sampling}

\begin{figure}[h]
	\centering
	\includegraphics[width=.67\linewidth]{$HOME/Dropbox/hsf/pic/destat/sample5}
	
		\caption{Sampling frame}\note{Source: \citet[p. 93]{Gonick1993Cartoon}}
\end{figure}

Random sampling is a sampling procedure by which each member of a population has an equal chance of
being included in the sample. Random sampling ensures a representative sample. There are several types of
random sampling. In simple random sampling, not only each item in the population but each sample has an
equal probability of being picked. In systematic sampling, items are selected from the population at uniform
intervals of time, order, or space (as in picking every one-hundredth name from a telephone directory).
Systematic sampling can be biased easily, such as, for example, when the amount of household garbage is
measured on Mondays (which includes the weekend garbage). In stratified and cluster sampling, the population is divided into strata (such as age groups) and clusters (such as blocks of a city) and then a proportionate
number of elements is picked at random from each stratum and cluster. Stratified sampling is used when the
variations within each stratum are small in relation to the variations between strata. Cluster sampling is used
when the opposite is the case. In what follows, we assume simple random sampling. Sampling can be from a
finite population (as in picking cards from a deck without replacement) or from an infinite population (as in
picking parts produced by a continuous process or cards from a deck with replacement).

\begin{figure}[h]
	\centering
	\includegraphics[width=.67\linewidth]{$HOME/Dropbox/hsf/pic/destat/sample4}
	\caption{Real world sampling}\note{Source: \citet[p. 92]{Gonick1993Cartoon}}
	
\end{figure}



\subsection{Simple random sample}

\begin{figure}[h]
	\centering
	\includegraphics[width=.67\linewidth]{$HOME/Dropbox/hsf/pic/destat/sample3}
	\caption{Simple random sample}\note{Source: \citet[p. 92]{Gonick1993Cartoon}}
	
\end{figure}



In statistics, a \textbf{simple random sample} is a subset of individuals (a sample) chosen from a larger set (a population). Each individual is chosen randomly and entirely by chance, such that each individual has the same probability of being chosen at any stage during the sampling process, and each subset of k individuals has the same probability of being chosen for the sample as any other subset of k individuals.
%Source: \citep{Wikipedia2020Trade}

\boxb{
	The simple random sample has two important properties:
	\begin{enumerate}[a)]
		\item \textbf{UNBIASED:} Each unit has the same chance of being chosen.\item \textbf{INDEPENDENCE:} Selection of one unit has no influence on the selection of other units.\end{enumerate} 
}



\exex{Examples of sample errors}{
	
	Watch \tv \textit{Sampling error and variation} \url{https://creativemaths.net/videos/video-variation-sources/}
\\
	and read \readsmall the following examples:
	
	\subsubsection{Example 1:} You have been hired by the National Election Commission to
	examine how the American people feel about the fairness of the voting procedures in the U.S. Who will you ask?
	
	It is not practical to ask every single American how he or she feels about the
	fairness of the voting procedures. Instead, we query a relatively small number of
	Americans, and draw inferences about the entire country from their responses. The
	Americans actually queried constitute our sample of the larger population of all
	Americans. The mathematical procedures whereby we convert information about
	the sample into intelligent guesses about the population fall under the rubric of
	inferential statistics.
	A sample is typically a small subset of the population. In the case of voting
	attitudes, we would sample a few thousand Americans drawn from the hundreds of
	millions that make up the country. In choosing a sample, it is therefore crucial that
	it not over-represent one kind of citizen at the expense of others. For example,
	something would be wrong with our sample if it happened to be made up entirely
	of Florida residents. If the sample held only Floridians, it could not be used to infer
	the attitudes of other Americans. The same problem would arise if the sample were
	comprised only of Republicans. \textbf{Inferential statistics are based on the assumption
		that sampling is random}. We trust a random sample to represent different segments
	of society in close to the appropriate proportions (provided the sample is large
	enough; see below).
	
	\subsubsection{Example 2:} We are interested in examining how many math classes have
	been taken on average by current graduating seniors at American colleges
	and universities during their four years in school. Whereas our population in
	the last example included all US citizens, now it involves just the graduating
	seniors throughout the country. This is still a large set since there are
	thousands of colleges and universities, each enrolling many students. It would be
	prohibitively costly to examine the transcript of every college senior. We
	therefore take a sample of college seniors and then make inferences to the
	entire population based on what we find. To make the sample, we might first
	choose some public and private colleges and universities across the United
	States. Then we might sample 50 students from each of these institutions.
	Suppose that the average number of math classes taken by the people in our
	sample were 3.2. Then we might speculate that 3.2 approximates the number
	we would find if we had the resources to examine every senior in the entire
	population. But we must \textbf{be careful about the possibility that our sample is
		non-representative of the population}. Perhaps we chose an overabundance of
	math majors, or chose too many technical institutions that have heavy math
	requirements. Such bad sampling makes our sample unrepresentative of the
	population of all seniors.
	To solidify your understanding of sampling bias, consider the following
	example. Try to identify the population and the sample, and then reflect on
	whether the sample is likely to yield the information desired.
	
	\subsubsection{Example 3:} A substitute teacher wants to know how students in the class
	did on their last test. The teacher asks the 10 students sitting in the front row
	to state their latest test score. He concludes from their report that the class
	did extremely well. What is the sample? What is the population? Can you
	identify any problems with choosing the sample in the way that the teacher
	did?
	
	In Example 3, the population consists of all students in the class. The sample is
	made up of just the 10 students sitting in the front row. \textbf{The sample is not likely to
		be representative of the population}. Those who sit in the front row tend to be more
	interested in the class and tend to perform higher on tests. Hence, the sample may
	perform at a higher level than the population.
	
	\subsubsection{Example 4:} A coach is interested in how many cartwheels the average
	college freshmen at his university can do. Eight volunteers from the
	freshman class step forward. After observing their performance, the coach
	concludes that college freshmen can do an average of 16 cartwheels in a row
	without stopping.
	
	In Example 4, the population is the class of all freshmen at the coach's university.
	The sample is composed of the 8 volunteers. The sample is poorly chosen because
	\textbf{volunteers are more likely to be able to do cartwheels} than the average freshman;
	people who can't do cartwheels probably did not volunteer! In the example, we are
	also not told of the gender of the volunteers. Were they all women, for example?
	That might affect the outcome, contributing to the non-representative nature of the
	sample.
	%
	\subsubsection{Example 5:} 
	Sometimes it is not feasible to build a sample using simple random sampling. To
	see the problem, consider the fact that both Dallas and Houston are competing to
	be hosts of the 2012 Olympics. Imagine that you are hired to assess whether most
	Texans prefer Houston to Dallas as the host, or the reverse. Given the
	impracticality of obtaining the opinion of every single Texan, you must construct a
	sample of the Texas population. But now notice how difficult it would be to
	proceed by simple random sampling. For example, how will you contact those
	individuals who don’t vote and don’t have a phone? Even among people you find
	in the telephone book, how can you identify those who have just relocated to
	California (and had no reason to inform you of their move)? What do you do about
	the fact that since the beginning of the study, an additional 4,212 people took up
	residence in the state of Texas? As you can see, it is sometimes very difficult to
	develop a truly random procedure.
	
%	A research scientist is interested in studying the experiences of
	%	twins raised together versus those raised apart. She obtains a list of twins
	%	from the National Twin Registry, and selects two subsets of individuals for
	%	her study. First, she chooses all those in the registry whose last name begins
	%	with Z. Then she turns to all those whose last name begins with B. Because
	%	there are so many names that start with B, however, our researcher decides
	%	to incorporate only every other name into her sample. Finally, she mails out
	%	a survey and compares characteristics of twins raised apart versus together.
	%
	%In Example 5, the population consists of all twins recorded in the National Twin
	%Registry. It is important that the researcher only make statistical generalizations to
	%the twins on this list, not to all twins in the nation or world. That is, the National
	%Twin Registry may not be representative of all twins. Even if inferences are limited
	%to the Registry, a number of problems affect the sampling procedure we described.
	%For instance, choosing only twins whose last names begin with Z does not give
	%every individual an equal chance of being selected into the sample. Moreover, such
	%a procedure risks over-representing ethnic groups with many surnames that begin
	%with Z. There are other reasons why choosing just the Z's may bias the sample.
	%Perhaps such people are more patient than average because they often find
	%themselves at the end of the line! The same problem occurs with choosing twins
	%whose last name begins with B. An additional problem for the B's is that the
	%``every-other-one'' procedure disallowed adjacent names on the B part of the list
	%from being both selected. Just this defect alone means the sample was not formed
	%through simple random sampling.
}

\subsection{Other sampling methods}


\subsubsection{Systematic sampling}
Systematic sampling (a.k.a. interval sampling) relies on arranging the study population according to some ordering scheme and then selecting elements at regular intervals through that ordered list. Systematic sampling involves a random start and then proceeds with the selection of every k$^{th}$ element from then onwards. 
\begin{center}\includegraphics[width=.7\linewidth]{$HOME/Dropbox/hsf/pic/destat/sample8}
	
	Source: \citet[p. 92]{Gonick1993Cartoon}\end{center}


\subsubsection{Accidental sampling / opportunity sampling / convenience sampling}
s a type of nonprobability sampling which involves the sample being drawn from that part of the population which is close to hand. That is, a population is selected because it is readily available and convenient.
\begin{center}\includegraphics[width=.7\linewidth]{$HOME/Dropbox/hsf/pic/destat/sample8a}
	
	Source: \citet[p. 92]{Gonick1993Cartoon}\end{center}


\subsubsection{Stratified sampling}

\begin{figure}[h]
\centering
\includegraphics[width=.7\linewidth]{$HOME/Dropbox/hsf/pic/destat/sample6}
		\caption{The population of pickles}
		\note{Source: \citet[p. 92]{Gonick1993Cartoon}}
		
\end{figure}

Since simple random sampling often does not ensure a representative sample, a
sampling method called stratified random sampling is sometimes used to make the
sample more representative of the population. This method can be used if the
population has a number of distinct ``strata'' or groups. In stratified sampling, you
first identify members of your sample who belong to each group. Then you
randomly sample from each of those subgroups in such a way that the sizes of the
subgroups in the sample are proportional to their sizes in the population.
Let's take an example: Suppose you were interested in views of capital
punishment at an urban university. You have the time and resources to interview
200 students. The student body is diverse with respect to age; many older people
work during the day and enroll in night courses (average age is 39), while younger
students generally enroll in day classes (average age of 19). It is possible that night
students have different views about capital punishment than day students. If 70\%
of the students were day students, it makes sense to ensure that 70\% of the sample
consisted of day students. Thus, your sample of 200 students would consist of 140
day students and 60 night students. The proportion of day students in the sample
and in the population (the entire university) would be the same. Inferences to the
entire population of students at the university would therefore be more secure.


\subsubsection{Cluster sampling}
Sometimes it is more cost-effective to select respondents in groups (clusters) of similar respondents. Sampling is often clustered by geography, or by time periods.
\begin{center}\includegraphics[width=.7\linewidth]{$HOME/Dropbox/hsf/pic/destat/sample7}
	
	Source: \citet[p. 92]{Gonick1993Cartoon}\end{center}


\subsection{Random assignment}

In experimental research, populations are often hypothetical. For example, in an
experiment comparing the effectiveness of a new anti-depressant drug with a
placebo, there is no actual population of individuals taking the drug. In this case, a
specified population of people with some degree of depression is defined and a
random sample is taken from this population. The sample is then randomly divided
into two groups; one group is assigned to the treatment condition (drug) and the
other group is assigned to the control condition (placebo). This random division of
the sample into two groups is called random assignment. \textbf{Random assignment is
critical for the validity of an experiment}. For example, consider the bias that could
be introduced if the first 20 subjects to show up at the experiment were assigned to
the experimental group and the second 20 subjects were assigned to the control
group. It is possible that subjects who show up late tend to be more depressed than
those who show up early, thus making the experimental group less depressed than
the control group even before the treatment was administered.
In experimental research of this kind, failure to assign subjects randomly to
groups is generally more serious than having a non-random sample. Failure to
randomize (the former error) invalidates the experimental findings. A non-random
sample (the latter error) simply restricts the generalizability of the results.



\section{Sample size matters}

 The sample size is an important feature of any empirical study in which the goal is to make inferences about a population from a sample. In practice, the sample size used in a study is usually determined based on the cost, time, or convenience of collecting the data, and the need for it to offer sufficient statistical power. 
 
Recall that the definition of a random sample is a sample in which every member of the population has an equal chance of being selected. This means that the \textbf{sampling procedure} rather than the \textbf{results} of the procedure define what it means for a sample to be random. Random samples, especially if the sample size is small, are not necessarily representative of the entire population. 

Larger sample sizes generally lead to increased precision when estimating unknown parameters. For example, if we wish to know the proportion of a certain species of fish that is infected with a pathogen, we would generally have a more precise estimate of this proportion if we sampled and examined 200 rather than 100 fish. Several fundamental facts of mathematical statistics describe this phenomenon, including the law of large numbers and the central limit theorem (see \autoref{sec:slt}). 

%For example, if a random sample of 20 subjects were taken from a population with an equal number
%of males and females, there would be a nontrivial probability (0.06) that 70\% or	
%more of the sample would be female. 
%Such a sample would not be representative, although it would be drawn randomly. Only a large sample size
%makes it likely that our sample is close to representative of the population. For this
%reason, inferential statistics take into account the sample size when generalizing
%results from samples to populations. 
%In later sections of the module, you'll see what kinds of mathematical techniques ensure this sensitivity to sample size.


