\chapter{The tidyverse universe}\label{ch:tidyverse}

\section{Introduction}
\boxx{
	Required reading:
	\itex{
		\item \citet[ch. 2]{Grolemund2018R}. See:  \url{https://r4ds.had.co.nz/workflow-basics.html}`
	}
Required exercise: 
	\itex{
		\item My swirl learning module `tidyverse'. See \pageref{exe:swirl-tidyverse}.
}
}

\begin{figure}[h]
	\centering
	\includegraphics[width=\linewidth]{../../../pic/Rstudio/tidyverse}
	\caption{The tidyverse universe}
	\label{fig:tidyverseuniverse}
\end{figure}

If you want to work with \R, I guess you should get known to the \rtext{tidyverse} package. 
The tidyverse is an opinionated collection of \R packages designed for data science. All packages share an underlying design philosophy, grammar, and data structures. 
The core packages are ggplot2, dplyr, tidyr, readr, purrr, tibble, stringr, and forcats, which provide functionality to model, transform, and visualize data. An additional 12 packages assist the core. As of November 2018, the tidyverse package and some of its individual packages comprise 5 out of the top 10 most downloaded R packages. The tidyverse is the subject of multiple books and papers.

Install the complete tidyverse with:

\begin{rblock1}
	install.packages("tidyverse")
\end{rblock1}

See how the tidyverse makes data science faster, easier and more fun with the book \textit{R for Data Science} from \cite{Grolemund2018R}. Read it online, buy the book or try another resource from the community, see e.g.: \websmall\url{www.tidyverse.org}.
On YouTube you find tons of material that exemplifies how to use \R. 

\section{The pipe operator}
\boxx{
	Required exercise: 
	\itex{
		\item Exercise `The pipe operator and others' on \autoref{app:pipe}. Also see  \url{https://github.com/hubchev/courses/blob/main/rmd/pipe-operator.Rmd}.
	}
}

\section{Data manipulation with \rtext{dplyr}}

\boxx{
	Required reading:
	\itex{
		\item \citet[ch. 3]{Grolemund2018R}. See:  \url{https://r4ds.had.co.nz/transform.html}
	}
}

\citet[ch. 3]{Grolemund2018R} is about the five key dplyr functions who allow you to solve the vast majority of your data-manipulation
challenges:
\itex{
	\item Pick observations by their values (\rtext{filter()}).
	\item Reorder the rows (\rtext{arrange()}).
	\item Pick variables by their names (\rtext{select()}).
	\item Create new variables with functions of existing variables
(\rtext{mutate()}).
	\item Collapse many values down to a single summary (\rtext{summarize()}).
}
These functions can all be used in conjunction with \rtext{group\_by()} , which
changes the scope of each function from operating on the entire
dataset to operating on it group-by-group. 

All functions work similarly:
\enux{
	\item The first argument is a data frame.
	\item The subsequent arguments describe what to do with the data
frame.
	\item The result is a new data frame.
}


\section{Using graphs and visualizing data}\label{ch:graphs}

\boxx{
	Required readings: 
	\itex{
		\item \textit{Using Graphs and Visualising Data} by Oliver \cite{Kirchkamp2018Using}: \url{https://www.kirchkamp.de/oekonometrie/pdf/gra-p.pdf}
		\item \citet[ch. 1]{Grolemund2018R}: \textit{Data Visualization with ggplot2}, See: \url{https://r4ds.had.co.nz/data-visualisation.html}
	}
	Required exercises:
	\itex{
		\item The \textit{Convergence} exercise found in \autoref{sec:convergence} and here: \url{https://github.com/hubchev/courses/blob/main/scr/convergence.R}
		\item The \textit{Regression analysis presentation} that can be found in the Appendix of these notes on pages \pageref{sec:regress_lecture}f and here: \url{https://htmlpreview.github.io/?https://github.com/hubchev/courses/blob/main/rmd/regress_lecture.html}
	}
}

\exex{What makes a graph ugly?}{
	Discuss what are the features of a good and bad graphical visualization of data. What do you think about the following graph?
	% TODO: \usepackage{graphicx} required
	\begin{center}
		\includegraphics[width=0.7\linewidth]{../../../pic/worst95}
	\end{center}
}
