\chapter{The tidyverse universe}\label{ch:tidyverse}

\section{Introduction}
\boxx{
	Required reading:
	\itex{
		\item \citet[ch. 2]{Grolemund2018R}. See:  \url{https://r4ds.had.co.nz/workflow-basics.html}`
	}
Required exercise: 
	\itex{
		\item My swirl learning module `tidyverse'. See \pageref{exe:swirl-tidyverse}.
}
}

\begin{figure}[h]
	\centering
	\includegraphics[width=\linewidth]{../../../pic/Rstudio/tidyverse}
	\caption{The tidyverse universe}
	\label{fig:tidyverseuniverse}
\end{figure}

If you want to work with \R, I guess you should get known to the \rtext{tidyverse} package. 
The tidyverse is an opinionated collection of \R packages designed for data science. All packages share an underlying design philosophy, grammar, and data structures. 
The core packages are ggplot2, dplyr, tidyr, readr, purrr, tibble, stringr, and forcats, which provide functionality to model, transform, and visualize data. An additional 12 packages assist the core. As of November 2018, the tidyverse package and some of its individual packages comprise 5 out of the top 10 most downloaded R packages. The tidyverse is the subject of multiple books and papers.

Install the complete tidyverse with:

\begin{rblock1}
	install.packages("tidyverse")
\end{rblock1}

See how the tidyverse makes data science faster, easier and more fun with the book \textit{R for Data Science} from \cite{Grolemund2018R}. Read it online, buy the book or try another resource from the community, see e.g.: \websmall\url{www.tidyverse.org}.
On YouTube you find tons of material that exemplifies how to use \R. 

\section{The pipe operator}
\boxx{
	Required exercise: 
	\itex{
		\item Exercise `The pipe operator and others' on \autoref{app:pipe}. Also see  \url{https://htmlpreview.github.io/?https://github.com/hubchev/courses/blob/main/rmd/pipe-operator.html}.
	}
}

\section{Data manipulation with \rtext{dplyr}}

\boxx{
	Required reading:
	\itex{
		\item \citet[ch. 3]{Grolemund2018R}. See:  \url{https://r4ds.had.co.nz/transform.html}
	}
}

\citet[ch. 3]{Grolemund2018R} is about the five key dplyr functions who allow you to solve the vast majority of your data-manipulation
challenges:
\itex{
	\item Pick observations by their values (\rtext{filter()}).
	\item Reorder the rows (\rtext{arrange()}).
	\item Pick variables by their names (\rtext{select()}).
	\item Create new variables with functions of existing variables
(\rtext{mutate()}).
	\item Collapse many values down to a single summary (\rtext{summarise()}).
}
These functions can all be used in conjunction with \rtext{group\_by()} , which
changes the scope of each function from operating on the entire
dataset to operating on it group-by-group. 

All functions work similarly:
\enux{
	\item The first argument is a data frame.
	\item The subsequent arguments describe what to do with the data
frame.
	\item The result is a new data frame.
}

\exex{dplyr cheatsheets}{
	Here \url{https://www.rstudio.com/resources/cheatsheets/} you find some cheatsheets on the \texttt{dplyr} package. Assign the following graphical sketches to one of the four functions mentioned above that are part of the \texttt{dplyr} package.
	
\includegraphics[width=.18\linewidth]{/home/sthu/Dropbox/hsf/pic/dsb/arrange.png}

\includegraphics[width=.2\linewidth]{/home/sthu/Dropbox/hsf/pic/dsb/mutate.png}

\includegraphics[width=.2\linewidth]{/home/sthu/Dropbox/hsf/pic/dsb/select.png}

\includegraphics[width=.2\linewidth]{/home/sthu/Dropbox/hsf/pic/dsb/filter.png}

\includegraphics[width=.2\linewidth]{/home/sthu/Dropbox/hsf/pic/dsb/summarise.png}

}


\exex{Subsetting}{
	\enux{
		\item Load the following packages: \rtext{tidyverse}, \rtext{dplyr}, and \rtext{tibble}.
		\item Check to see if you have the \rtext{mtcars} dataset by entering the command \rtext{mtcars}.
		\item Save the \rtext{mtcars} dataset in an object named \rtext{cars}.
		\item What class is \rtext{cars}?
		\item How many observations (rows) and variables (columns) are in the mtcars dataset?
		\item Rename mpg in cars to \rtext{MPG}. Use \rtext{rename()}.
		\item Convert the column names of cars to all upper case. Use \rtext{rename\_all}, and the \rtext{toupper} command.
		\item Convert the rownames of \rtext{cars} to a column called \rtext{car} using \rtext{rownames\_to\_column}. 
		\item Subset the columns from \rtext{cars} that end in "\rtext{p}" and call it \rtext{pvars} using \rtext{ends\_with()}.
		\item Create a subset cars that only contains the columns: \rtext{wt}, \rtext{qsec}, and \rtext{hp} and assign this object to \rtext{carsSub}.  (Use \rtext{select()}.)
		\item What are the dimensions of \rtext{carsSub}? (Use \rtext{dim()}.)
		\item Convert the column names of \rtext{carsSub} to all upper case. Use \rtext{rename\_all()}, and \rtext{toupper()} (or \rtext{colnames()}).
		\item Subset the rows of cars that get more than 20 miles per gallon (\rtext{mpg}) of fuel efficiency. How many are there? (Use \rtext{filter()}.)
		\item Subset the rows that get less than 16 miles per gallon (\rtext{mpg}) of fuel efficiency and have more than 100 horsepower (\rtext{hp}). How many are there? (Use \rtext{filter()} and the pipe operator.)
		\item Create a subset of the cars data that only contains the columns: \rtext{wt}, \rtext{qsec}, and \rtext{hp} for cars with 8 cylinders (\rtext{cyl}) and reassign this object to \rtext{carsSub}. What are the dimensions of this dataset? Don't use the pipe operator.
		\item Create a subset of the cars data that only contains the columns: \rtext{wt}, \rtext{qsec}, and \rtext{hp} for cars with 8 cylinders (\rtext{cyl}) and reassign this object to \rtext{carsSub2}. Use the pipe operator.
		\item Re-order the rows of \rtext{carsSub} by weight (\rtext{wt}) in increasing order. (Use \rtext{arrange()}.)
		\item Create a new variable in carsSub called wt2, which is equal to $wt^2$, using \rtext{mutate()} and piping \rtext{\%>\%}.
	}
	
	\boxx{
		Please find solutions here: 
		\url{https://raw.githubusercontent.com/hubchev/courses/main/scr/exe_subset.R}
	}
}
